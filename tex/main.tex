\documentclass{article}
\usepackage{amsmath,amsthm,amssymb}
\usepackage{mathtext}
\usepackage[T2A]{fontenc}
\usepackage[utf8]{inputenc}
\usepackage[english]{babel}
\usepackage{graphicx}
\usepackage{hyperref}


\title{Russian address elements classification using artificial neural networks}
\author{Anton Reshetnikov}
\date{May 2024}



\begin{document}
\maketitle
\begin{abstract}
    This documents provides result of research of token classification task applied to russian place addresses.
    The task is to recognize elements of address like region, area, city, territory, street.
    The source code available at
     \url{https://github.com/qwazer/ruaddress-elements-classification}.
\end{abstract}



\section{Introduction}


Logistics companies have the task of checking and normalizing the address against the incoming string.
For example, there is an input string with the address ``Москва, Абрат, 1'' (in Russian).
It is necessary to select address-forming elements, check against the database,
that such an address actually exists and return the status of the address and its normalized representation.

Traditionally, IT systems that solve this problem using a rules-based approach (text tokenization, checks using a set of regular expressions).
The task is to explore the possibilities of solving the same problem using modern deep learning models.

\subsection{Team}

\textbf{Anton Reshetnikov} - researcher



\section{Related Work}
\label{sec:related}
\cite{makarov2020algo} provides overview of traditional algorithms used for address elements recognition.
The particular solution for address element recognition described in ~\cite{habr2020gar}
%todo overview of international attempts

\section{Model Description}

The most modern deep learning models based on Transformer architecture described in the ``Attention Is All You Need'' article~\cite{vaswani2023attention}.
There are several models adapted for russian language exists.
For detail see this overview article:~\cite{zmitrovich2024family}.

For the purpose of current research \href{https://huggingface.co/cointegrated/rubert-tiny2}{cointegrated/rubert-tiny2} model selected.
Reasons of such choice are:
\begin{enumerate}
    \item It based on Transformer architecture;
    \item The model is focused on Russian;
    \item Small size, which make it suitable for fast experiments.
\end{enumerate}


\section{Dataset}

\subsection{Datasource}
The dataset based on freely available ``State address register'' distributed at \url{https://fias.nalog.ru/} by Federal Taxation Service of the Russian Federation.

Subset of ``State address register'' related to address-forming elements used to build the dataset.
Only address elements related to Administrative division (Административно-территориальное деление) was selected.

\subsection{Token classes}

Dictionary ``Element type'' used as token classes for token classification task.
Values of dictionary are:
\begin{enumerate}
    \item REGION
    \item REGION\_TYPE
    \item AREA
    \item AREA\_TYPE
    \item TERRITORY
    \item TERRITORY\_TYPE
    \item CITY
    \item CITY\_TYPE
    \item STREET
    \item STREET\_TYPE
\end{enumerate}

\subsection{Mapping table}
Address elements levels of ``State address register'' mapped to custom dictionary ``Element type'' with next mapping table:


\begin{center}
    \begin{tabular}{| l | l | l | }
        \hline
        State address register level (in Russian) & Element type & Example  \\
        \hline
        Субъект & REGION & Омская область  \\
        Административный район & AREA & Любинский р-н   \\
        Город & CITY &    \\
        Населенный пункт & CITY &  поселок Камышловский \\
        Элемент планировочной структуры & TERRITORY & Территория СНТ Сибзаводовец-2  \\
        Элемент улично-дорожной сети & STREET & 7-я аллея \\
        Земельный участок & not used & \\
        Здание (сооружение) & not used &  \\
        Помещение & not used & \\
        Помещения в пределах помещения & not used  &\\
        Машино-место & not used  &\\
        \hline
    \end{tabular}
\end{center}

\subsection{Dataset structure}

The idea is to generate own dataset called ``Ruaddress'' from ``State address register'' datasource.
The ``Ruaddress'' dataset has 2 columns, described in the next table:

\begin{center}
    \begin{tabular}{| l | p{6cm} | l | }
        \hline
        Column name & tokens & classes  \\
        Description & list of token words & list of class codes  \\
        Example & [Вологодская, Область, Грязовецкий, Район, Вохтога, Рабочий, поселок] &  [1, 2, 3, 4, 7, 8, 8] \\
        \hline
    \end{tabular}
\end{center}



\subsection{Augmenation}

A place address can be presented in a various forms.
To generate address form from address elements the augmentation procedure is used.
The procedure receive address-forming elements for a place, then apply series of transformation and output 2 lists: tokens and related token classes.

\subsubsection{Word level augmentation}
\subsubsection{Char level augmentation}

\cite{martynov2023augmentation}

%TODO describe dataset Augmenation

\subsection{Final dataset}
%TODO describe dataset metrics

\section{Experiments}

\subsection{Metrics}
Standard metrics for classification task are used: Precision, Recall, F1.

\subsection{Experiment Setup}
Secondly, you need to describe the design of your experiment, e.g. how many runs there were, how the data split was done. The important details of your model, like hyper-parameters used in the experiments, and so on.

\subsection{Baselines}
Another important feature is that you could provide here the description of some simple approaches for your problem, like logistic regression over TF-IDF embedding for text classification. The baselines are needed is there is no previous art on the problem you are presenting.

\section{Results}
In this section, you need to list and describe the achieved results. It is crucial to have the results of the experiments for the other approaches. This is needed to be able to compare your results with some competitors. Most preferably, you should provide some references with results on the same problem.

Almost inevitably the results are presented as a table, but it is also possible to have a graph, i.e. a figure.

You need also to provide an interpretation of the presented results, to describe some features. E.g. your approach shows higher results on the short texts or by one metric instead of another.

Also in this section, you could provide some results for your model inference. The samples could be found in Tab.~\ref{tab:output}.

\begin{table}[!tbh]
    \centering
    \begin{tabular}{|c|}
\hline
Это пример вывода вашей модели на русском.\\
This is a sample output of your model in English.
\\
\hline
    \end{tabular}
    \caption{Output samples.}
    \label{tab:output}
\end{table}

\section{Conclusion}
In this section, you need to describe all the work in short: what you have done and what has been achieved. E.g. you have collected a dataset, made a markup for it and developed a model showing the best results compared to other models.

\bibliographystyle{apalike}
\bibliography{lit}
\end{document}